\documentclass[a4paper]{ltjsarticle}
%preamble.tex

%LaTeXエンジン
\usepackage{luatexja}

%フォント
%\usepackage[ipaex]{luatexja-preset}

%図表
\usepackage{graphicx}
\usepackage{tikz}
\usepackage{multirow}
\usepackage{float}
\usepackage{wrapfig}

%数学
\usepackage{mathtools}
\usepackage{amsmath}

%科学
\usepackage{physics}
\usepackage{siunitx}
\usepackage[version=4]{mhchem}

%リンク
\usepackage{url}

%ハイパーリンク
\usepackage[unicode,hidelinks,pdfusetitle]{hyperref}

%余白
\usepackage[margin=12truemm]{geometry}

%枠付き
\usepackage{ascmac}
\usepackage{fancybox}

%%%%%%%%%%%%%%%%%%%%%%%%%%%%%%%%%%%%%%%%%%%%%%%%%
\begin{document}

\title{北海道大学理学部地球惑星科学科 オープンキャンパス\\クレーター形成実験}
\date{\today}
\maketitle

%%%%%%%%%%%%%%%%%%%%%%%%%%%%%%%%%%%%%%%%%%%%%%%%%
\section{実験結果}
\begin{figure}[H]
    \centering
    \includegraphics[height=10cm]{./figures/graph.pdf}
    \caption{形成されたクレーターの画像}
\end{figure}


\end{document}

%%%%%%%%%%%%%%%%%%%%%%%%%%%%%%%%%%%%%%%%%%%%%%%%%
%\begin{figure}[H]
%    \centering
%    \includegraphics[height=]{}
%    \caption{}
%    \label{}
%\end{figure}

%\begin{equation}
%    \label{}
%\end{equation}

%\begin{table}[H]
%    \caption{}
%    \label{}
%    \centering
%    \begin{tabular}{}
%        \hline
%        \multicolumn{}{}{}
%    \end{tabular}
%\end{table}

%\begin{itemize}
%    \item
%    \item
%    \item
%\end{itemize}

%\begin{tikzpicture}[scale=0.95]
%    \filldraw [fill=gray, draw=gray] (0,4.5) circle[radius=3cm];
%    \filldraw [fill=gray, draw=gray] (-3,4.5) rectangle (3,7.5);
%    \filldraw [fill=gray!25, draw=black] (0,10) circle[radius=2.5cm];
%    \draw (-3,4.5) -- (-3,12) -- (-6,12) -- (-6,0) -- (6,0) -- (6,12) -- (3,12) -- (3,4.5);
%    \draw (-3.5,4.5) -- (-3.5,11.5) -- (-5.5,11.5) -- (-5.5,0.5) -- (5.5,0.5) -- (5.5,11.5) -- (3.5,11.5) -- (3.5,4.5);
%    \draw (-3,4.5) arc (180:360:3cm);
%    \draw (-3.5,4.5) arc (180:360:3.5cm);
%    \node (a) at (-4.5,12.5) [above] {透明ガラスデュワー};
%    \draw [->] (a) -- (-4.5,12.1);
%    \node (b) at (2.5,12.5) [above] {風船};
%    \draw [->] (b) -- (1.8,11.8);
%\end{tikzpicture}

%\begin{enumerate}
%    \item
%  \end{enumerate}
